\chapter{Background}
\label{ch:background}
This chapter provides a general description and minimum introduction to the various concepts and tools (of both mathematical and computational nature) used in this thesis. The following sections will hopefully be self-contained and will act as a springboard for understanding the results presented in this thesis and their relevance in the broader network research paradigm. Wherever possible references are provided for further consultation.

\section{Complex Networks}

\section{Statistical Methods}

\section{Data Fitting Methods}

\section{Graph Tools}

\section{Programming Languages}
The bulk of the work in this thesis was done in Python 2.7\footnote{Python programming language: \href{https://www.python.org/}{www.python.org}}. Apart from the core pre-installed modules, the essential trio of scientific programming -- \lstinline|SciPy|, \lstinline|NumPy| and \lstinline|matplotlib|, were used. The \lstinline|igraph|, \lstinline|graph_tool| and PATHPY packages mentioned in the previous sections were also run in Python. The pseudocode provided in the later chapters are also based on Python syntax.

The gHypE package was used in R\footnote{R Project: \href{https://www.r-project.org/}{www.r-project.org}}. The typesetting of this manuscript was done using \LaTeX. 